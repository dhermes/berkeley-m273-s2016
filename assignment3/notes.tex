\documentclass[letterpaper,10pt]{article}

\usepackage[margin=1in]{geometry}

\usepackage{amsthm,amssymb,amsmath,tikz}
\usetikzlibrary{decorations.markings}

\usepackage{embedfile}
\embedfile{\jobname.tex}

\usepackage{hyperref}
\hypersetup{colorlinks=true,urlcolor=blue}

\usepackage{fancyhdr}
\pagestyle{fancy}
\lhead{}
\rhead{2D Conservation Law Schemes}

\renewcommand{\headrulewidth}{0pt}
\renewcommand{\qed}{\(\blacksquare\)}

\begin{document}
Note that for a vector valued \(\mathbf{F}\left(\mathbf{x}\right)\) and
a scalar \(v\left(\mathbf{x}\right)\) we have a sort of product rule for
divergence:
\[\nabla \cdot \left(v\mathbf{F}\right) = \nabla v \cdot \mathbf{F} +
v \left(\nabla \cdot \mathbf{F}\right).\]
Thus a weak solution \(u\) of
\[u_t + \nabla \cdot \mathbf{F} = 0\]
must satisfy
\[\int_{\Omega} u_t v \, dV =
-\int_{\Omega} v \left(\nabla \cdot \mathbf{F}\right) \, dV =
\int_{\Omega} \nabla v \cdot \mathbf{F} -
\nabla \cdot \left(v\mathbf{F}\right) \, dV =
\int_{\Omega} \nabla v \cdot \mathbf{F} \, dV -
\int_{\partial \Omega} \left(v\mathbf{F}\right) \cdot \mathbf{n} \, dS.\]

We are considering the ``rotating flux'' function
\[\mathbf{F} = u \left[ \begin{array}{c} -y \\ x
\end{array}\right]\]
on the domain \((x, y) \in \left[-1, 1\right]^2\) with initial
condition
\[u(x, y, 0) = \frac{1}{2\pi \cdot \frac{1}{8}^2} \exp\left(
\frac{\left(x - \frac{1}{2}\right)^2 + y^2}{\frac{1}{8}^2}\right)\]
and boundary conditions
\[u(x, -1, t) = u(x, 1, t) = u(-1, y, t) = u(1, y, t) = 0.\]

Now our test functions must satisfy
\[\int_{\Omega} u_t v \, dV =
\int_{\Omega} u\left(-y v_x + x v_y\right) \, dV -
\int_{\partial \Omega} uv \left(-y n_x + x n_y\right) \, dS.\]
To use DG to solve this problem, our test functions and components
of \(u\) will be degree \(p\) polynomials hence
\(u\left(-y v_x + x v_y\right)\) is degree \(2p\) and
\(uv \left(-y n_x + x n_y\right)\) is degree \(2p + 1\).

To make this concrete, we'll consider \(p = 1\). The quadrature
rule
\[\int_T f\left(\mathbf{x}\right) \, d\mathbf{x} \approx
\frac{\left|T\right|}{3}
\left[f\left(\frac{C + v_0}{2}\right) +
f\left(\frac{C + v_1}{2}\right) +
f\left(\frac{C + v_2}{2}\right)\right]\]
is exact for quadratics (here \(C = \frac{v_0 + v_1 + v_2}{3}\) is
the centroid of \(T\)).

On each directed edge of \(\partial T\), we have a parameterization
\(\gamma(s) = v_i + \frac{s + 1}{2}(v_{i + 1} - v_i)\). Writing
\(v_{i + 1} - v_i = \left[ \begin{array}{c} \Delta x_i \\ \Delta y_i
\end{array}\right]\) we have an outward\footnote{one can
check that the cross product
\(\mathbf{n}_i \times (v_{i + 1} - v_i) = \frac{\Delta x_i^2 +
\Delta y_i^2}{\left|v_{i + 1} - v_i\right|}\) points in the positive
\(z\)-direction} normal given by
\(\mathbf{n}_i = \frac{1}{\left|v_{i + 1} - v_i\right|} \left[
\begin{array}{c} \Delta y_i \\ -\Delta x_i \end{array}\right]\) hence
\[\int_{\gamma} \left[ \begin{array}{c} f \\ g \end{array}\right]
\cdot \mathbf{n}_i \, dS =
\int_{-1}^1 \frac{f\left(\gamma(s)\right) \Delta y_i -
g\left(\gamma(s)\right) \Delta x_i}{
\left|v_{i + 1} - v_i\right|} \left|\gamma'(s)\right| \, ds =
\int_{-1}^1 \frac{f\left(\gamma(s)\right) \Delta y_i -
g\left(\gamma(s)\right) \Delta x_i}{2} \, ds.\]
So if \(f, g\) are cubics, the Gaussian quadrature
\begin{align*}
\int_{-1}^1 \frac{f\left(\gamma(s)\right) \Delta y_i -
g\left(\gamma(s)\right) \Delta x_i}{2} \, ds &\approx
\frac{f\left(\gamma\left(-1 / \sqrt{3}\right)\right) \Delta y_i -
g\left(\gamma\left(-1 / \sqrt{3}\right)\right) \Delta x_i}{2} \\
&+ \frac{f\left(\gamma\left(1 / \sqrt{3}\right)\right) \Delta y_i -
g\left(\gamma\left(1 / \sqrt{3}\right)\right) \Delta x_i}{2}
\end{align*}
is exact. For our given function we have
\[2\int_{\gamma} vu \left[ \begin{array}{c} -y \\ x \end{array}\right]
\cdot \mathbf{n} \, dS \approx
-uv \left(\gamma\left(-\frac{1}{\sqrt{3}}\right)
\cdot \Delta v_i\right)
- uv \left(\gamma\left(\frac{1}{\sqrt{3}}\right)
\cdot \Delta v_i\right).\]

On a given triangular element \(T\) with (ordered, local) vertices
\(v_0, v_1, v_2\) we have a map
\[R(x, y) = (1 - x - y) v_0 + x v_1 + y v_2\]
from the reference triangle \(T_0\) to \(T\), with this
\[\int_{R(T_0)} f \, dV = \int_{T_0} f(R(\mathbf{x}))
\left|\det J\right| \,d \mathbf{x}.\]
Note that \(\left|\det J\right| = 2\left|T\right|\). For identity
functions \(\varphi_i\) such that \(\varphi_i(v_j) = \delta_{ij}\),
we must have \(\varphi_i\left(R\left(v_j^{(0)}\right)\right) =
\delta_{ij}\) (where \(v_j^{(0)}\) are the nodes of the reference
triangle) hence we must have \(\varphi_0\left(R(x, y)\right) = 1 - x - y\),
\(\varphi_1\left(R(x, y)\right) = x\) and
\(\varphi_2\left(R(x, y)\right) = y\) by uniqueness of these ``hat''
functions.

For \(u_t = \dot{u}_0 \varphi_0 + \dot{u}_1 \varphi_1 + \dot{u}_2 \varphi_2\)
(again in local indices) we have
\[\int_T u_t \left[ \begin{array}{c}
\varphi_0 \\ \varphi_1 \\ \varphi_2 \end{array}\right] \, dV =
\frac{\left|T\right|}{12} \left[ \begin{array}{c c c}
2 & 1 & 1 \\
1 & 2 & 1 \\
1 & 1 & 2 \end{array}\right] \left[ \begin{array}{c}
\dot{u}_0 \\ \dot{u}_1 \\ \dot{u}_2 \end{array}\right] \Longrightarrow
M_T^{-1} = \frac{3}{\left|T\right|} \left[ \begin{array}{c c c}
 3 & -1 & -1 \\
-1 &  3 & -1 \\
-1 & -1 &  3 \end{array}\right].\]
One can show that in the \(p = 1\) case
\[\left[ \begin{array}{c c}
\frac{\partial \varphi_i}{\partial x} &
\frac{\partial \varphi_i}{\partial y}
\end{array}\right] =
\frac{1}{2\left|T\right|} \left[ \begin{array}{c c}
-\Delta y_1 & \Delta x_1 \\
-\Delta y_2 & \Delta x_2 \\
-\Delta y_0 & \Delta x_0
\end{array}\right]\]
we can write the quadrature points as
\[\left[ \begin{array}{c c c}
q_0 & q_1 & q_2
\end{array}\right]
= \frac{1}{6} \left[ \begin{array}{c c c}
x_0 & x_1 & x_2 \\
y_0 & y_1 & y_2
\end{array}\right] \left[ \begin{array}{c c c}
4 & 1 & 1 \\
1 & 4 & 1 \\
1 & 1 & 4
\end{array}\right]\]
hence computing
\[g_i = -y \frac{\partial \varphi_i}{\partial x} +
x \frac{\partial \varphi_i}{\partial y}\]
at each of these three points can be accomplished via
\begin{align*}
G = \left[ \begin{array}{c} g_i\left(q_j\right) \end{array}\right]
&= \frac{1}{12\left|T\right|} \left[ \begin{array}{c c}
-\Delta y_1 & \Delta x_1 \\
-\Delta y_2 & \Delta x_2 \\
-\Delta y_0 & \Delta x_0
\end{array}\right] \left[ \begin{array}{c c c}
-y_0 & -y_1 & -y_2 \\
x_0 & x_1 & x_2
\end{array}\right] \left[ \begin{array}{c c c}
4 & 1 & 1 \\
1 & 4 & 1 \\
1 & 1 & 4
\end{array}\right] \\
&= \frac{1}{12\left|T\right|} \left[ \begin{array}{c c c}
\Delta v_1 & \Delta v_2 & \Delta v_0
\end{array}\right]^T \left[ \begin{array}{c c c}
v_0 & v_1 & v_2
\end{array}\right] \left[ \begin{array}{c c c}
4 & 1 & 1 \\
1 & 4 & 1 \\
1 & 1 & 4
\end{array}\right]
\end{align*}
We combine this with
\[Q = \left[ \begin{array}{c} \varphi_i\left(q_j\right)
\end{array}\right] =
\frac{1}{6} \left[ \begin{array}{c c c}
4 & 1 & 1 \\
1 & 4 & 1 \\
1 & 1 & 4
\end{array}\right]\]
to compute
\[\int_{T} \varphi_j \left(-y \frac{\partial \varphi_i}{\partial x} +
x \frac{\partial \varphi_i}{\partial y}\right) \, dV =
\frac{\left|T\right|}{3} \sum_{k = 0}^2 \varphi_j(q_k) g_i(q_k).\]
This corresponds to taking the dot product of row \(j\) of \(Q\)
with row \(i\) of \(G\). But, due to the
symmetry of \(Q\) these \(9\) values actually occur in
\[K = \frac{1}{6} \left[ \begin{array}{c c c}
\Delta v_1 & \Delta v_2 & \Delta v_0
\end{array}\right]^T \left[ \begin{array}{c c c}
v_0 & v_1 & v_2
\end{array}\right] Q^2\]
Thus far, we have
\[\left|T\right| M \left[ \begin{array}{c}
\dot{u}_0 \\ \dot{u}_1 \\ \dot{u}_2 \end{array}\right] =
K \left[ \begin{array}{c}
u_0 \\ u_1 \\ u_2 \end{array}\right] -
\int_{\partial T} \left(v\mathbf{F}\right) \cdot \mathbf{n} \, dS.\]
To handle the final integral, we need to utilize a sort of
``upwind'' condition. We consider
\[\operatorname{sign}\left(\beta(x, y) \cdot \mathbf{n}_i\right) =
\operatorname{sign}\left(\left[ \begin{array}{c}
-y \\ x \end{array}\right] \cdot \left[ \begin{array}{c}
\Delta y_i \\ -\Delta x_i \end{array}\right]\right) =
-\operatorname{sign}\left(x \Delta x_i + y \Delta y_i\right) =
-\operatorname{sign}\left(\gamma(s) \cdot \Delta v_i\right).\]
If \(\operatorname{sign}\left(\beta(x, y) \cdot \mathbf{n}_i\right)
> 0\), we use \(u_i\) and \(u_{i + 1}\) to parameterize our line.
If not, then we use \(u_i^{(i)}\) and \(u_{i + 1}^{(i)}\):

\begin{center}
\begin{tikzpicture}[scale=3.0]
\draw[blue, fill=blue] (0.04, 0.04) circle (0.01cm);
\draw[blue, fill=blue] (0.93, 0.03) circle (0.01cm);
\draw[blue, fill=blue] (0.03, 0.93) circle (0.01cm);
\draw (0.1, 0.1) node{\scriptsize{\(u_0\)}};
\draw (0.8, 0.07) node{\scriptsize{\(u_1\)}};
\draw (0.07, 0.84) node{\scriptsize{\(u_2\)}};

\draw (0, 0) -- (1, 0);
\draw (1, 0) -- (0, 1);
\draw (0, 1) -- (0, 0);

\draw[dashed] (1, 0) -- (1/2, -0.866);
\draw[dashed] (1/2, -0.866) -- (0, 0);
\draw[blue, fill=blue] (0.05, -0.03) circle (0.01cm);
\draw[blue, fill=blue] (0.93, -0.03) circle (0.01cm);
\draw (0.18, -0.1) node{\scriptsize{\(u_0^{(0)}\)}};
\draw (0.8, -0.08) node{\scriptsize{\(u_1^{(0)}\)}};

\draw[dashed] (1, 0) -- (1.366, 1.366);
\draw[dashed] (1.366, 1.366) -- (0, 1);
\draw[blue, fill=blue] (0.98, 0.07) circle (0.01cm);
\draw[blue, fill=blue] (0.07, 0.98) circle (0.01cm);
\draw (0.95, 0.18) node{\scriptsize{\(u_1^{(1)}\)}};
\draw (0.2, 0.97) node{\scriptsize{\(u_2^{(1)}\)}};

\draw[dashed] (0, 1) -- (-0.866, 1/2);
\draw[dashed] (-0.866, 1/2) -- (0, 0);
\draw[blue, fill=blue] (-0.04, 0.07) circle (0.01cm);
\draw[blue, fill=blue] (-0.03, 0.93) circle (0.01cm);
\draw (-0.09, 0.83) node{\scriptsize{\(u_2^{(2)}\)}};
\draw (-0.09, 0.17) node{\scriptsize{\(u_0^{(2)}\)}};
\end{tikzpicture}
\end{center}

For example, against the (local) test function \(\varphi_i\)
\[2 \int_{\left(\partial T\right)_0} \varphi_i u
\left(\left[ \begin{array}{c} -y \\ x \end{array}\right]
\cdot \mathbf{n}_0\right) \, dS =
\left.\varphi_i u^{\pm} \right|_{s=-\frac{1}{\sqrt{3}}}
\left(-\gamma_{-} \cdot \Delta v_0\right) +
\left.\varphi_i u^{\pm} \right|_{s=\frac{1}{\sqrt{3}}}
\left(-\gamma_{+} \cdot \Delta v_0\right).\]
Each of \(\varphi_i\) and \(u\) are lines, hence linear in
\(s\), this allows us to simplify
\[\left.\varphi_0\right|_{\left(\partial T\right)_0} =
\frac{1 - s}{2}, \quad
\left.\varphi_1\right|_{\left(\partial T\right)_0} =
\frac{1 + s}{2}, \quad
\left.\varphi_2\right|_{\left(\partial T\right)_0} = 0, \quad
\left.u^{\pm}\right|_{\left(\partial T\right)_0} =
\frac{1 - s}{2} u_0^{\pm} + \frac{1 + s}{2} u_1^{\pm}\]
so that
\begin{align*}
2 \int_{\left(\partial T\right)_i} \varphi_i u
\left(\left[ \begin{array}{c} -y \\ x \end{array}\right]
\cdot \mathbf{n}_i\right) \, dS &=
\left.\varphi_i u \right|_{s=-\frac{1}{\sqrt{3}}}
\left(-\gamma_{-} \cdot \Delta v_i\right) +
\left.\varphi_i u \right|_{s=\frac{1}{\sqrt{3}}}
\left(-\gamma_{+} \cdot \Delta v_i\right) \\
&= \kappa_{+} \left(\kappa_{+} u_i + \kappa_{-} u_{i + 1}\right)
\left(-\gamma_{-} \cdot \Delta v_i\right) +
\kappa_{-} \left(\kappa_{-} u_i + \kappa_{+} u_{i + 1}\right)
\left(-\gamma_{+} \cdot \Delta v_i\right) \\
2 \int_{\left(\partial T\right)_i} \varphi_{i + 1} u
\left(\left[ \begin{array}{c} -y \\ x \end{array}\right]
\cdot \mathbf{n}_i\right) \, dS &=
\left.\varphi_{i + 1} u \right|_{s=-\frac{1}{\sqrt{3}}}
\left(-\gamma_{-} \cdot \Delta v_i\right) +
\left.\varphi_{i + 1} u \right|_{s=\frac{1}{\sqrt{3}}}
\left(-\gamma_{+} \cdot \Delta v_i\right) \\
&= \kappa_{-} \left(\kappa_{+} u_i + \kappa_{-} u_{i + 1}\right)
\left(-\gamma_{-} \cdot \Delta v_i\right) +
\kappa_{+} \left(\kappa_{-} u_i + \kappa_{+} u_{i + 1}\right)
\left(-\gamma_{+} \cdot \Delta v_i\right) \\
2 \int_{\left(\partial T\right)_i} \varphi_{i + 2} u
\left(\left[ \begin{array}{c} -y \\ x \end{array}\right]
\cdot \mathbf{n}_i\right) \, dS &= 0
\end{align*}
where \(\kappa_{\pm} = \frac{1 \pm \frac{1}{\sqrt{3}}}{2}\).

\section{Other Stuff}

For a DG scheme \(p = 1\), we consider the reference triangle \(T\) and
\[\left.u\right|_T = u_0 \varphi_0 + u_1 \varphi_1 + u_2 \varphi_2, \quad
\left.u_t\right|_T = \dot{u}_0 \varphi_0 + \dot{u}_1 \varphi_1 +
\dot{u}_2 \varphi_2\]
where \(\varphi_0 = 1 - x - y\), \(\varphi_1 = x\) and \(\varphi_2 = y\).
\[\int_{T} u_t \varphi_i \, dV =
\dot{u}_0 \int_{T} \varphi_i \varphi_0 \, dV +
\dot{u}_1 \int_{T} \varphi_i \varphi_1 \, dV +
\dot{u}_2 \int_{T} \varphi_i \varphi_2 \, dV\]
where these coefficients are given by the mass matrix
\[M = \frac{1}{24} \left[ \begin{array}{c c c}
2 & 1 & 1 \\
1 & 2 & 1 \\
1 & 1 & 2 \end{array}\right].\]

We are considering the function
\[\mathbf{F} = u \left[ \begin{array}{c} -y \\ x
\end{array}\right]\]
hence \(\nabla v \cdot \mathbf{F} = v_x(-yu) + v_y(xu) =
u(x v_y - y v_x)\) and we need to evaluate
\[\int_{T} \nabla \varphi_i \cdot \mathbf{F} \, dV =
\sum_{j = 0}^2 u_j \int_{T} \varphi_j \left(
x \frac{\partial \varphi_i}{\partial y} -
y \frac{\partial \varphi_i}{\partial x}\right) \, dV\]
which gives the ``stiffness'' matrix
\[K = \frac{1}{24} \left[ \begin{array}{c c c}
0 & -1 & 1 \\
-1 & -1 & -2 \\
1 & 2 & 1 \end{array}\right].\]
As it turns out, each column of \(K\) is an eigenvector of \(M\) with
the same eigenvalue, hence \(M^{-1} K\) is easy to compute
(this ``matters'' for future computations).

At this point we have
\[\frac{1}{24} \left[ \begin{array}{c c c}
2 & 1 & 1 \\
1 & 2 & 1 \\
1 & 1 & 2 \end{array}\right] \left[ \begin{array}{c}
\dot{u}_0 \\ \dot{u}_1 \\ \dot{u}_2 \end{array}\right] =
\frac{1}{24} \left[ \begin{array}{c c c}
0 & -1 & 1 \\
-1 & -1 & -2 \\
1 & 2 & 1 \end{array}\right] \left[ \begin{array}{c}
u_0 \\ u_1 \\ u_2 \end{array}\right] -
\int_{\partial T} \left(
\left[ \begin{array}{c}
\varphi_0 \\ \varphi_1 \\ \varphi_2 \end{array}\right]
\mathbf{F}\right) \cdot \mathbf{n} \, dS.\]

Along the reference triangle \(T\), we have outward normals given by
\(\mathbf{n}_0 = \left[ \begin{array}{c}
0 \\ -1 \end{array}\right]\), \(\mathbf{n}_1 = \frac{1}{\sqrt{2}}
\left[ \begin{array}{c} 1 \\ 1 \end{array}\right]\) and
\(\mathbf{n}_2 = \left[ \begin{array}{c}
-1 \\ 0 \end{array}\right]\).

%% H/T: http://tex.stackexchange.com/questions/39278/tikz-arrowheads-in-the-center
\tikzset{middlearrow/.style={
        decoration={markings,
            mark= at position 0.5 with {\arrow{#1}} ,
        },
        postaction={decorate}
    }
}

\begin{center}
\begin{tikzpicture}[scale=3.0]
\draw[fill=black] (0, 0) circle (0.02cm);
\draw[fill=black] (1, 0) circle (0.02cm);
\draw[fill=black] (0, 1) circle (0.02cm);

\draw[middlearrow={latex}] (0, 0) -- (1, 0);
\draw[middlearrow={latex}] (1, 0) -- (0, 1);
\draw[middlearrow={latex}] (0, 1) -- (0, 0);

\draw (-0.1, -0.1) node{\scriptsize{\(v_0\)}};
\draw (1.1, -0.1) node{\scriptsize{\(v_1\)}};
\draw (-0.1, 1.1) node{\scriptsize{\(v_2\)}};

\draw (0.5, -0.2) node{\scriptsize{\(\gamma_0 =
\left[ \begin{array}{c} s \\ 0 \end{array}\right]\)}};
\draw (0.8, 0.6) node{\scriptsize{\(\gamma_1 =
\left[ \begin{array}{c} 1 - s \\ s \end{array}\right]\)}};
\draw (-0.35, 0.5) node{\scriptsize{\(\gamma_2 =
\left[ \begin{array}{c} 0 \\ 1 - s \end{array}\right]\)}};
\end{tikzpicture}
\end{center}
The final integral (noting that \(\varphi_i \mathbf{F} =
\varphi_i u \left[ \begin{array}{c} -y \\ x \end{array}\right]\))
becomes
\begin{align*}
\int_{\partial T} \left(\varphi_i
\mathbf{F}\right) \cdot \mathbf{n} \, dS &=
\int_0^1 \left(\varphi_i \mathbf{F}\right) \cdot \mathbf{n}_0
\left|\gamma_0'(s)\right| \, ds + \cdots \\
&= \int_0^1 - \varphi_i x(s) u \, ds +
\sqrt{2} \int_0^1 \varphi_i u \frac{x(s) - y(s)}{\sqrt{2}} \, ds +
\int_0^1 \varphi_i y(s) u \, ds.
\end{align*}
and computing this integrals we see
\begin{align*}
\int_{\partial T} \left(
\left[ \begin{array}{c}
\varphi_0 \\ \varphi_1 \\ \varphi_2 \end{array}\right]
\mathbf{F}\right) \cdot \mathbf{n} \, dS &=
\left(\frac{1}{12} \left[ \begin{array}{c c c}
-1 & -1 & 0 \\
-1 & -3 & 0 \\
 0 &  0 & 0
\end{array}\right] +
\frac{1}{6} \left[ \begin{array}{c c c}
0 & 0 &  0 \\
0 & 1 &  0 \\
0 & 0 & -1
\end{array}\right] +
\frac{1}{12} \left[ \begin{array}{c c c}
1 & 0 & 1 \\
0 & 0 & 0 \\
1 & 0 & 3
\end{array}\right]\right)
\left[ \begin{array}{c}
u_0 \\ u_1 \\ u_2 \end{array}\right] \\
&= \left(G_0 + G_1 + G_2\right)
\left[ \begin{array}{c}
u_0 \\ u_1 \\ u_2 \end{array}\right].
\end{align*}
Rather than use the values \(u_0, u_1, u_2\) in \(T\), we instead
``reach across'' the edges of \(T\) as a sort of upwind condition:

\begin{center}
\begin{tikzpicture}[scale=3.0]
\draw[blue, fill=blue] (0.04, 0.04) circle (0.01cm);
\draw[blue, fill=blue] (0.93, 0.03) circle (0.01cm);
\draw[blue, fill=blue] (0.03, 0.93) circle (0.01cm);
\draw (0.1, 0.1) node{\scriptsize{\(u_0\)}};
\draw (0.8, 0.07) node{\scriptsize{\(u_1\)}};
\draw (0.07, 0.84) node{\scriptsize{\(u_2\)}};

\draw (0, 0) -- (1, 0);
\draw (1, 0) -- (0, 1);
\draw (0, 1) -- (0, 0);

\draw[dashed] (1, 0) -- (1/2, -0.866);
\draw[dashed] (1/2, -0.866) -- (0, 0);
\draw[blue, fill=blue] (0.05, -0.03) circle (0.01cm);
\draw[blue, fill=blue] (0.93, -0.03) circle (0.01cm);
\draw (0.18, -0.1) node{\scriptsize{\(u_0^{(0)}\)}};
\draw (0.8, -0.08) node{\scriptsize{\(u_1^{(0)}\)}};

\draw[dashed] (1, 0) -- (1.366, 1.366);
\draw[dashed] (1.366, 1.366) -- (0, 1);
\draw[blue, fill=blue] (0.98, 0.07) circle (0.01cm);
\draw[blue, fill=blue] (0.07, 0.98) circle (0.01cm);
\draw (0.95, 0.18) node{\scriptsize{\(u_1^{(1)}\)}};
\draw (0.2, 0.97) node{\scriptsize{\(u_2^{(1)}\)}};

\draw[dashed] (0, 1) -- (-0.866, 1/2);
\draw[dashed] (-0.866, 1/2) -- (0, 0);
\draw[blue, fill=blue] (-0.04, 0.07) circle (0.01cm);
\draw[blue, fill=blue] (-0.03, 0.93) circle (0.01cm);
\draw (-0.09, 0.83) node{\scriptsize{\(u_2^{(2)}\)}};
\draw (-0.09, 0.17) node{\scriptsize{\(u_0^{(2)}\)}};
\end{tikzpicture}
\end{center}
All together, the update condition becomes
\[M \left[ \begin{array}{c}
\dot{u}_0 \\ \dot{u}_1 \\ \dot{u}_2 \end{array}\right] = K
\left[ \begin{array}{c}
u_0 \\ u_1 \\ u_2 \end{array}\right] -
G_0 \left[ \begin{array}{c}
u_0^{(0)} \\ u_1^{(0)} \\ 0 \end{array}\right] -
G_1 \left[ \begin{array}{c}
0 \\ u_1^{(1)} \\ u_2^{(1)} \end{array}\right] -
G_2 \left[ \begin{array}{c}
u_0^{(2)} \\ 0 \\ u_2^{(2)} \end{array}\right].\]
In order to reduce the complexity of the computation, we produce here
\begin{align*}
M^{-1} K = 24K &= \left[ \begin{array}{c c c}
 0 & -1 &  1 \\
-1 & -1 & -2 \\
 1 &  2 &  1 \end{array}\right] \\
M^{-1} G_0 &= \left[ \begin{array}{c c c}
-1 &  0 & 0 \\
-1 & -4 & 0 \\
 1 &  2 & 0 \end{array}\right] \\
M^{-1} G_1 &= \left[ \begin{array}{c c c}
0 & -1 &  1 \\
0 &  3 &  1 \\
0 & -1 & -3 \end{array}\right] \\
M^{-1} G_2 &= \left[ \begin{array}{c c c}
 1 & 0 &  0 \\
-1 & 0 & -2 \\
 1 & 0 &  4 \end{array}\right].
\end{align*}
For boundary edges, where there is no other triangle ``across the edge'',
we use the boundary condition.

\end{document}
